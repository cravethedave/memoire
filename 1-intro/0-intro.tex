\documentclass[../Document.tex]{subfiles}
\graphicspath{{\subfix{../images/}}}

\begin{document}


% Dans l'introduction, on présente le problème étudié et les buts
% poursuivis. L'introduction permet de faire connaître le cadre de la
% recherche et d'en préciser le domaine d'application. Elle fournit
% les précisions nécessaires en ce qui concerne le contexte de
% réalisation de la recherche, l'approche envisagée, l'évolution de
% la réalisation. En fait, l'introduction présente au lecteur ce
% qu'il doit savoir pour comprendre la recherche et en connaître la
% portée.
\Chapter{INTRODUCTION}
\label{chap:introduction}

Drug discovery is a very time-consuming and costly endeavor due to its enormous design space
--- estimated to contain between $10^{23}$ and $10^{60}$ different molecules~\cite{molecule-space} --- and to the lengthy and failure-fraught process of bringing a product to market.
Automated molecule design is nowadays a vital part of drug discovery and material science, with computational approaches coming from deep generative models and combinatorial search methods~\cite{du2022molgensurvey}.
It aims to extract from this huge design space the most likely candidates according to some desired properties.
Even among these, only a few may lead to a usable product after extensive testing.

\gls{smiles}, a one-dimensional encoding of molecules, is one of the standards commonly used by this research community.
It lends itself well to techniques used for \gls{nlp}, such as sequential generative neural models.
\glspl{llm} in particular have come to the forefront of popular attention as impressive tools for generating text.
However, this generation isn't limited to purely human languages as we can train the model on another text format, such as \gls{smiles}, and get a model capable of generating molecules.

\gls{smiles} also lends itself very well to \gls{cp}. \gls{cp} seems like a natural approach to molecule discovery since it allows hard constraints to be placed which could ensure only valid molecules are generated.
Using a \gls{cfg} and a few additional constraints could allow us to describe valid \gls{smiles} strings in a \gls{cp} model.
We also believe it may be possible to model desirable molecular properties using \gls{cp}, this would allow our model to restrict its search even further.
This allows us to explore the huge design space of possible molecules while adding constraints in order to restrict that space to suitable candidates.

This \gls{cp} approach, which excels at imposing hard rules and long-term structure while lacking the informed decision making that trained models gain from the dataset used, could allow us to answer one of the issues with sequence models in \gls{ml}.
Often times, these models struggle to exhibit long-term structure, stemming in part from the token-by-token nature of the prediction process used to generate a sequence.
In other words, these models do not explicitly learn the hard rules that determine validity nor desirability and merely mimic what was observed.

While this problem can be addressed at training, by changing what the model is trained on such that it can better learn the structure, we wish to introduce a \gls{cpbp} model at inference time (generation). This should enforce the presence of the desired structure, which is critical if it is mandatory~\cite{deutsch2019general,lee2019gradient}.
In other words, this technique could allow us to target properties and structures that the model was never trained to generate while still maintaining advantages of the original model.

This guarantees that a generated molecule will respect the desired structure, which is critical if it is mandatory~\cite{deutsch2019general,lee2019gradient}.

This combined model is of more interest, however, when we wish to impose constraints that were not featured in the training dataset of the model. This allows the satisfaction of these new constraints while avoiding the retraining of the model, which can be costly with larger models.

This combination of both techniques could lead to valid, realistic and property-constrained molecules. However, there is a balance to maintain as we do not wish to stray too far from what was featured in the training dataset in order to respect the imposed constraints. This is particularly difficult for long-term structure, which requires balancing foresight over many yet-to-be generated tokens and the immediacy of next-token predictions from the sequence model.



%%%%%%%%% Section CFG %%%%%%%%%
\subfile{1-cfg.tex}


%%%%%%%%% Section Chemistry %%%%%%%%%
\subfile{2-chemistry.tex}


%%%%%%%%% Section Constraint Programming %%%%%%%%%
\subfile{3-cp.tex}


%%%%%%%%% Section Natural Language Processing %%%%%%%%%
\subfile{4-nlp.tex}


\section{Problem Statement}
\label{sec:intro/problem}

As mentioned previously, drug discovery is both time-consuming and costly and automated drug discovery has been an important field of research to reduce these costs.
While \gls{ml} methods have been gaining a lot of popularity in the field, those techniques suffer from a lack of long-term structure.
To address this, \gls{cp} is a natural answer since it provides the lacking long-term structure.
\david{Cite works using CP could help}
However, while \gls{cp} is used in the domain, there isn't much work \david{I found none, but to be investigated further} relating to generating molecule candidates using \gls{cp}.

We believe that a \gls{cp} model would be beneficial and would reduce the number of invalid molecules generated.

More importantly however, a \gls{cp} model that generates valid molecules could then be used to target property-specific molecules using constraints to eliminate undesirable options.

The issue of using \gls{cp} for this problem is the size of the search space to explore. By using \gls{bp}, we believe that the search will be better guided towards a solution and require less backtracking. However, it remains to be seen if the added cost for the \gls{bp} increases the overall time to solve.

Finally, we believe that by combining a trained token-by-token generating \gls{ml} model with our \gls{cpbp} model, we might get molecules similar to what is being used today (molecules in datasets) while still maintaining the long-term structure of the \gls{cp} model.


\section{Research Questions}
\label{sec:intro/questions}
During our research we will answer the following questions:

\begin{enumerate}
    \item Can we use \acrshort{cp} to model valid molecules in a one-dimensional encoding?
    \item Can we use \acrshort{cp} to model desirable molecular properties in \acrshort{smiles} molecules?
    \item Can \acrlong{bp} be used to better guide a solver towards a solution?
    \item How can we combine a \acrshort{cp} model with a \acrshort{nlp} model to improve the realism of generated sequences and is it an effective method?
\end{enumerate}

\section{Thesis Outline}
\label{sec:intro/outline}
The rest of this thesis is organized in the following chapters:

\begin{itemize}
    \item Chapter \ref{chap:background} goes over the necessary concepts to understand the rest of the paper.
    \item Chapter \ref{chap:lit-review} provides a general overview of the different techniques currently in use.
    \item Chapter \ref{chap:cp-validity} presents our base model as well as the methods used to model valid molecules as per our first research question.
    \item Chapter \ref{chap:cp-properties} expands on the previous section and introduces ways to model molecular properties using \acrshort{cp}. This section addresses our second research question.
    \item Chapter \ref{chap:gpt+cp} details how we combine our \acrshort{cp} model 
    with a \acrshort{nlp} model.
    \item Chapter \ref{chap:conclusion} goes over the paper's contributions, its limitations and potential ways to improve this in future work.
\end{itemize}


\end{document}