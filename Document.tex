% !TeX root = Document.tex
% !TeX program = pdflatex
% !TeX glossary = makeglossaries

\documentclass[letterpaper,12pt,oneside,final]{book}

\usepackage[hidelinks]{hyperref}

%% Support des acronymes. / Support for acronyms.
\usepackage[acronym]{glossaries}
\makeglossaries

\newacronym{ai}{AI}{Artificial Intelligence}
\newacronym{bert}{BERT}{Bidirectional Encoder Representations from Transformers}
\newacronym{bp}{BP}{Belief Propagation}
\newacronym{cfg}{CFG}{Context-Free Grammar}
\newacronym{cp}{CP}{Constraint Programming}
\newacronym{csp}{CSP}{Constraint Satisfaction Problem}
\newacronym{cpbp}{CPBP}{Constraint Programming with Belief Propagation}
\newacronym{dfs}{DFS}{Depth-First Search}
\newacronym{gpt}{GPT}{Generative Pre-trained Transformer}
\newacronym{inchi}{InChI}{International Chemical Identifier}
\newacronym{iupac}{IUPAC}{International Union of Pure and Applied Chemistry}
\newacronym{llm}{LLM}{Large Language Model}
\newacronym{ml}{ML}{Machine Learning}
\newacronym{nlp}{NLP}{Natural Language Processing}
\newacronym{nn}{NN}{Neural Network}
\newacronym{rnn}{RNN}{Recursive Neural Network}
\newacronym{selfies}{SELFIES}{SELF-referencIng Embedded Strings}
\newacronym{smiles}{SMILES}{Simplified Molecular Input Line Entry System}

%%
%%  Gabarit bilingue de mémoire de maîtrise ou thèse de doctorat.
%%  Bilingual template for dissertations and theses @ Polytechnique Montreal.

%%  Normalement, il n'est pas nécessaire de modifier ce document
%%  sauf pour établir le langage (français ou anglais) et pour changer les noms des fichiers à inclure.
%%  Usually, this document needs to be modified only to set up the language (French or English) and to change the names of the files to include.
%%
%%  Version: 2023-01-20
%%
%% Accepte les caractères accentués dans le document (UTF-8).
%% Supports accented characters in the document (UTF-8).

\newcommand*\circled[1]{\tikz[baseline=(char.base)]{\node[shape=circle,draw,inner sep=2pt](char) {#1};}}
\newcommand{\constraint}[1]{\mbox{\sc #1}}
\newcommand{\extensional}{\constraint{Table}}
\newcommand{\shortTable}{\constraint{ShortTable}}
\newcommand{\somme}{\constraint{Sum}}
\newcommand{\alldiff}{\constraint{AllDifferent}}
\newcommand{\regular}{\constraint{Regular}}
\newcommand{\costregular}{\constraint{CostRegular}}
\newcommand{\grammar}{\constraint{Cfg}}
\newcommand{\mdd}{\constraint{Mdd}}
\newcommand{\exactly}{\constraint{Exactly}}
\newcommand{\atmost}{\constraint{AtMost}}
\newcommand{\gcc}{\constraint{Gcc}}
\newcommand{\among}{\constraint{Among}}
\newcommand{\element}{\constraint{Element}}
\newcommand{\inverse}{\constraint{Inverse}}
\newcommand{\lexless}{\constraint{LexLess}}
\newcommand{\alldiffMod}{\constraint{AllDifferent\_modulo}}
\newcommand{\maxMod}{\constraint{Maximum\_modulo}}
\newcommand{\amongMod}{\constraint{Among\_modulo}}
\newcommand{\cp}{{\sc cp}}
\newcommand{\bp}{{\sc bp}}
\newcommand{\csp}{{\sc csp}}
\newcommand{\sat}{{\sc sat}}
\newcommand{\mip}{{\sc mip}}
\newcommand{\ai}{{\sc ai}}

\usepackage{todonotes}
\newcommand{\david}[1]{\todo[inline,color=cyan]{{\bf Do this:} #1}}
\newcommand{\gilles}[1]{\todo[inline,color=cyan]{{\bf Gilles commented:} #1}}
\newcommand{\askGilles}[1]{\todo[inline,color=green]{{\bf @Gilles, I have a question:} #1}}
\usepackage{booktabs}
\usepackage{cancel}
\usepackage{longtable}
\usepackage{pgfplots}
\usetikzlibrary{matrix,fit,automata}

\makeatletter
\newcommand{\myparagraph}[1]{\paragraph{#1}\mbox{}\\}
\newcommand{\ie}{\emph{i.e.}\@ifnextchar.{\!\@gobble}{ }}
\newcommand{\eg}{\emph{e.g.}\@ifnextchar.{\!\@gobble}{ }}
\newcommand{\etc}{etc\@ifnextchar.{}{.\@}}

\def\bstctlcite{\@ifnextchar[{\@bstctlcite}{\@bstctlcite[@auxout]}}
\def\@bstctlcite[#1]#2{\@bsphack
 \@for\@citeb:=#2\do{%
   \edef\@citeb{\expandafter\@firstofone\@citeb}%
   \if@filesw\immediate\write\csname #1\endcsname{\string\citation{\@citeb}}\fi}%
 \@esphack}
\makeatother

%% LA COMMANDE SUIVANTE ÉTABLIT LE LANGAGE DE LA THÈSE : ÉCRIRE french POUR UNE THÈSE EN FRANÇAIS
%% THE NEXT COMMAND DETERMINES THE LANGUAGE OF THE THESIS: WRITE english FOR A THESIS IN ENGLISH
\newcommand\Langue{english}

\usepackage{ifthen}
\usepackage[utf8]{inputenc}
%%
%% Support pour l'anglais et le français (français par défaut).
%% Support for English and French (French by default).

%\usepackage[cyr]{aeguill}
\usepackage{lmodern}      % Police de caractères plus complète et généralement indistinguable visuellement de la police standard de LaTeX (Computer Modern). / A more complete and generally visually indistinguishable font from the standard LaTeX font (Computer Modern).
\usepackage[T1]{fontenc}  % Bon encodage des caractères pour qu'Acrobat Reader reconnaisse les accents et les ligatures telles que ffi. / Good character encoding so that Acrobat Reader recognizes accents and ligatures such as ffi.


\usepackage[english]{babel} 

%%
%% Charge le module d'affichage graphique. / Loads the graphics package.
\usepackage{graphicx}
\usepackage{epstopdf}  % Permet d'utiliser des .eps avec pdfLaTeX. / Allows using .eps with pdfLaTeX.
%%
%% Recherche des images dans les répertoires. / Search for images in folders.
\graphicspath{{./images/}{./dia/}{./gnuplot/}}
%%
%% Un float peut apparaître seulement après sa définition, jamais avant. / A float can appear only after its definition, never before.
\usepackage{flafter,placeins}
%%
%% Autres modules. / Other packages.
\usepackage{amsmath,color,soulutf8,longtable,colortbl,setspace,xspace,url,pdflscape,cite}

\onehalfspacing                % Interligne 1.5. / Line spacing = 1.5.
%%
%% Définition d'un style de page avec seulement le numéro de page à
%% droite. On s'assure aussi que le style de page par défaut soit
%% d'afficher le numéro de page en haut à droite. / Definition of a page 
%% style with only the page number on the right. We also make sure that the 
%% default page style is to display the page number at the top right.
\usepackage{fancyhdr}
\fancypagestyle{pagenumber}{\fancyhf{}\fancyhead[R]{\thepage}}
\renewcommand\headrulewidth{0pt}
\makeatletter
\let\ps@plain=\ps@pagenumber
\makeatother
%%
%% Module qui permet la création des bookmarks dans un fichier PDF. / Package that allows the creation of bookmarks in a PDF file.
%\usepackage[dvipdfm]{hyperref}
\usepackage{caption}  % Hyperlien vers la figure plutôt que son titre. / Hyperlink to the figure rather than its title.
\makeatletter
\providecommand*{\toclevel@compteur}{0}
\makeatother

%% Modules ajoutés (2022) / packages added (2022)
\usepackage{subcaption} % figures & sous figures / figures & subfigures
\usepackage{siunitx} % unites SI / SI units
\usepackage{amssymb} % autres symboles mathematiques / other mathematical symbols
\usepackage[bottom]{footmisc} % pour avoir les notes de bas de page en dessous des figures... / to have the footnotes below the figures
%\usepackage{listings} % Si on veut ajouter des lignes de codes dans le texte / If you want to add lines of code to the text


%%
%% Définitions spécifiques au format de rédaction de Poly.
%% Here we define the Poly formatting.
\RequirePackage[\Langue]{MemoireThese}
%%
%% Définitions spécifiques à l'étudiant.
%% Student-specific definitions.
%% -----------------------------------
%% ---> À MODIFIER PAR L'ETUDIANT / TO BE MODIFIED BY THE STUDENT <---
%% -----------------------------------
%%
%% Commandes qui affichent le titre du document, le nom de l'auteur, etc.
% Commands that display the document title, the author's name, etc.
\newcommand\monTitre{Modelling Valid and Desirable Molecules using Constraint Programming and Hybrid Machine Learning Models}
\newcommand\monPrenom{David}
\newcommand\monNom{Saikali}
\newcommand\monDepartement{Génie informatique et génie logiciel}  % Department
\newcommand\maDiscipline{Génie informatique}
\newcommand\monDiplome{M}        % (M)aîtrise ou (D)octorat / (M)aster or Ph(D)
\newcommand\anneeDepot{2025}    % Year
\newcommand\moisDepot{Juillet}       % Month
\newcommand\monSexe{M}           % "M" ou "F" = Gender
\newcommand\PageGarde{Y}         % "O" ou "N" = Yes or No
\newcommand\AnnexesPresentes{O}  % "O" ou "N". Indique si le document comprend des annexes. / If the thesis includes annexes = O; if it does not N = No.
\newcommand\mesMotsClef{Liste,de,mot-clés,séparés,par,des,virgules}
%%
%%  DEFINITION DU / OF JURY
%%
%%  Pour la définition du jury, les macros suivantes sont definies: / For the definition of the jury, the following macros are defined:
%%  \PresidentJury, \DirecteurRecherche, \CoDirecteurRecherche, \MembreJury, \MembreExterneJury
%%
%%  Toutes les macros prennent 3 paramètres: Sexe (M/F), Nom, Prénom
%%  All the macros have 3 parameters: Gender (M/F), Last name, First name
\newcommand\monJury{\PresidentJury{F}{Zouaq}{Amal}\\
\DirecteurRecherche{M}{Pesant}{Gilles}\\
% \CoDirecteurRecherche{F}{Couture}{Marie}\\
\MembreJury{M}{Rousseau}{Louis-Martin}}
% \MembreExterneJury{M}{Brown}{Joseph}}

\ifthenelse{\equal{\monDiplome}{M}}{
\newcommand\monSujet{Mémoire de maîtrise}
\newcommand\monDipl{Maîtrise ès sciences appliquées}
}{
\newcommand\monSujet{Thèse de doctorat}
\newcommand\monDipl{Philosophi\ae{} Doctor}
}
%%
%% Informations qui sont stockées dans un fichier PDF.
%% Information that is stored in a PDF file.
\hypersetup{
  pdftitle={\monTitre},
  pdfsubject={\monSujet},
  pdfauthor={\monPrenom{} \monNom},
  pdfkeywords={\mesMotsClef},
  bookmarksnumbered,
  pdfstartview={FitV},
  hidelinks,
  linktoc=all
}

%% Ajoute en 2022 (ajout des titres complets des tables et figure et alignement)
%% Added in 2022 (added full table and figure titles and alignment)
\usepackage[titles]{tocloft}
  \renewcommand{\cftchapleader}{\cftdotfill{\cftsecdotsep}} % dotted chapter leaders

\renewcommand\cfttabindent{0pt}
\renewcommand\cfttabnumwidth{7em}
\renewcommand\cfttabpresnum{\tablename\ }

\renewcommand\cftfigindent{0pt} 
\renewcommand\cftfignumwidth{7em}
\renewcommand\cftfigpresnum{\figurename\ }

\ifthenelse{\equal{\Langue}{english}}{
	\renewcommand\cftchapfont{CHAPTER }
    \renewcommand\cftchappagefont{}
}{
	\renewcommand\cftchapfont{CHAPITRE }
    \renewcommand\cftchappagefont{}
}
%

%%
%% Il y a un document par chapitre du mémoire ou thèse.
%% There is one document per chapter of the thesis or dissertation.

\usepackage{subfiles}

\begin{document}
\bstctlcite{IEEEexample:BSTcontrol}

%%
%% Page de titre du mémoire ou de la thèse.
%% Title page of the dissertation or thesis.
\frontmatter
%% Compte optionellement la page de garde dans la pagination.
%% Optionally counts the cover page in the pagination.
\ifthenelse{\equal{\PageGarde}{O}}{\addtocounter{page}{1}}{}
\thispagestyle{empty}%
\begin{center}%
\vspace*{\stretch{0.1}}
\textbf{POLYTECHNIQUE MONTRÉAL}\\
affiliée à l'Université de Montréal\\
\vspace*{\stretch{1}}
\textbf{\monTitre}\\
\vspace*{\stretch{1}}
\textbf{\MakeUppercase{\monPrenom~\monNom}}\\
Département de~{\monDepartement}\\
\vspace*{\stretch{1}}
\ifthenelse{\equal{\monDiplome}{M}}{Mémoire présenté}{Thèse présentée} en vue de l'obtention du diplôme de~\emph{\monDipl}\\
\maDiscipline\\
\vskip 0.4in
\moisDepot~\anneeDepot
\end{center}%
\vspace*{\stretch{1}}
\copyright~\monPrenom~\monNom, \anneeDepot.
%%
%% Identification des membres du jury.
%% Jury members.
\newpage\thispagestyle{empty}%
\begin{center}%

\vspace*{\stretch{0.1}}
\textbf{POLYTECHNIQUE MONTRÉAL}\\
affiliée à l'Université de Montréal\\
\vspace*{\stretch{2}}
Ce\ifthenelse{\equal{\monDiplome}{M}}{~mémoire intitulé}{tte thèse intitulée} :\\
\vspace*{\stretch{1}}
\textbf{\monTitre}\\
\vspace*{\stretch{1}}
présenté\ifthenelse{\equal{\monDiplome}{M}}{}{e}
par~\textbf{\mbox{\monPrenom~\MakeUppercase{\monNom}}}\\
en vue de l'obtention du diplôme de~\emph{\mbox{\monDipl}}\\
a été dûment accepté\ifthenelse{\equal{\monDiplome}{M}}{}{e} par le jury d'examen constitué de :\end{center}
\vspace*{\stretch{2}}
\monJury
%%
\pagestyle{pagenumber}%


% Dédicace du document
\subfile{0-pre/1-dedicace.tex}

% Remerciements / Acknowledments
\subfile{0-pre/2-thanks.tex}

% Résumé du sujet en français / Abstract in French
\subfile{0-pre/3.1-Resume.tex}

% Résumé du sujet en anglais / Abstract in English
\subfile{0-pre/3.2-Abstract.tex}

{\setlength{\parskip}{0pt}
%%
%% Table des matières
%% Table of contents
%\ifthenelse{\equal{\Langue}{english}}{
%	\renewcommand\contentsname{TABLE OF CONTENTS}
%}{
%	\renewcommand\contentsname{TABLE DES MATIÈRES}
%}
%\tableofcontents
\newpage
\ifthenelse{\equal{\Langue}{english}}{
	\begin{center}
	    \textbf{TABLE OF CONTENTS}
        \addtocontents{toc}{\protect{\pdfbookmark[0]{TABLE OF CONTENTS}{toc}}}
	\end{center}
}{
	\begin{center}
	    \textbf{TABLE DES MATIÈRES}
        \addtocontents{toc}{\protect{\pdfbookmark[0]{TABLE DES MATIÈRES}{toc}}}
	\end{center}
}
\makeatletter
\renewcommand{\tableofcontents}{%
  \@starttoc{toc}%
}
\makeatother
\makeatletter
% Ensure sections and subsections are properly indented
\renewcommand{\cftsecindent}{1.5em}
\renewcommand{\cftsubsecindent}{4em}
\renewcommand{\cftsubsubsecindent}{4.5em}

\makeatother
\tableofcontents
%%
%% Liste des tableaux
%% List of tables
\ifthenelse{\equal{\Langue}{english}}{
	\renewcommand\listtablename{LIST OF TABLES}
}{
	\renewcommand\listtablename{LISTE DES TABLEAUX}
}\listoftables
%%
%% Liste des figures
%% List of figures
\ifthenelse{\equal{\Langue}{english}}{
	\renewcommand\listfigurename{LIST OF FIGURES}
}{
	\renewcommand\listfigurename{LISTE DES FIGURES}
}\listoffigures
%%
%% Liste des annexes au besoin.
%% List of appendices, if needed.
}


% Liste des sigles et abréviations.
\subfile{0-pre/4-abrev.tex}


\ifthenelse{\equal{\AnnexesPresentes}{O}}{\listofappendices}{}


\mainmatter
% Introduction au sujet de recherche.
\subfile{1-intro/0-intro.tex}

% Background
\subfile{2-background/0-background.tex}

 % Revue de littérature / Literature review
\subfile{3-review/0-review.tex}

% Premier thème (Doctorat) ou "Détails de la Solution" (Maîtrise) / First topic (PhD) or "Details of the Solution" (Master's).
\subfile{4-cp/0-valid.tex}

% Second thème (Doctorat) ou "Résultats théoriques et expérimentaux" (Maîtrise) / Second theme (PhD) or "Theoretical and experimental results" (Master's)
\subfile{4-cp/1-properties.tex}

% Troisième thème (Doctorat) ou effacez ce fichier si vous êtes à la Maîtrise / Third topic (PhD) or delete this file if you are in the Master's program
\subfile{5-gptcp/0-gptcp.tex}

% Conclusion.
\subfile{6-conclusion/0-conclusion.tex}


%\backmatter
\ifthenelse{\equal{\Langue}{english}}{
	\renewcommand\bibname{REFERENCES}
	\bibliography{Document}
	\bibliographystyle{IEEEtran}			% Style bibliographique / Bibliography style 
}{
	\renewcommand\bibname{RÉFÉRENCES}
	\bibliography{Document}
	\bibliographystyle{IEEEtran-francais}    % Style bibliographique / Bibliography style 
}
%
\ifthenelse{\equal{\AnnexesPresentes}{O}}{
	\appendix%
	\newcommand{\Annexe}[1]{\annexe{#1}\setcounter{figure}{0}\setcounter{table}{0}\setcounter{footnote}{0}}%
	\include{9-Annexes}}
{}
\end{document}
