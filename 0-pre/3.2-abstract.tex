\documentclass[../Document.tex]{subfiles}
\graphicspath{{\subfix{../images/}}}

\begin{document}


%% Abstract
%%
%% Traduction anglaise fidèle et de qualité du résumé de la recherche écrit en français et non une traduction littérale. 
%%

\chapter*{ABSTRACT}\thispagestyle{headings}
\addcontentsline{toc}{compteur}{ABSTRACT}

Drug discovery is a very costly endeavor in both time and resources and, unfortunately, most possible molecules are not desirable.
Using automated techniques such as \acrshort{ml} has become standard to reduce the number of likely candidates or to target specific types of molecules.
However, these techniques often struggle to exhibit long term structure.

Among the standard formats used to encode molecules, \acrshort{smiles} is a widespread string representation that has gained traction in both \acrshort{ml} and chemistry circles.

We propose a constraint programming model showcasing the grammar constraint to express the design space of organic molecules using the \acrshort{smiles} notation.

We show how some common physicochemical properties --- such as molecular weight and lipophilicity --- and structural features can be expressed as constraints in the model.

We also address the lack of long term structure in \acrshort{ml} models by introducing our neurosymbolic framework \acrshort{geai-blanc}.
The learned probabilities of the sequence model are mixed in with the marginal probabilities from a constraint programming / belief propagation framework at inference time.
The next predicted token is then selected from the resulting probability distribution.
Experiments on this hybrid model show that we can achieve the post-training imposed structure without straying too much from the structure of the dataset learned during training.


\end{document}