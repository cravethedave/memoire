\documentclass[../Document.tex]{subfiles}
\graphicspath{{\subfix{../images/}}}

\begin{document}


%% Abstract
%%
%% Traduction anglaise fidèle et de qualité du résumé de la recherche écrit en français et non une traduction littérale. 
%%

\chapter*{ABSTRACT}\thispagestyle{headings}
\addcontentsline{toc}{compteur}{ABSTRACT}
%
\begin{otherlanguage}{english}

This thesis goes over our efforts to represent drug-like molecules using \acrlong{cp}.
\david{Ajoute une phrase de motivation, soit avant soit après.}
We also attempt to evaluate desirable properties to guide our results towards potentially useful molecules.
To try and improve the realism of generated molecules, we combine \acrlong{ml}, specifically \acrlong{nlp}, and \acrlong{cp}.
We use perplexity and the success rate to evaluate our model's quality. This allows us to weigh the different constraints against the information learned by the model.
If our work shows promise, we believe it could be useful in other domains to apply long-term structure in long sequence generation.


Written in English, the abstract is a brief summary similar to the previous
section {\selectlanguage{french}(Résumé)}. However, this section is not a
word for word translation of the abstract in French.

The abstract is a brief statement of the subject matter, objectives, research questions or hypotheses, experimental methods and analysis of results. It also presents the main research conclusions and their possible applications. In general, an abstract should not exceed three pages.

The abstract should provide an exact idea of the thesis or dissertation’s contents and it cannot be a simple enumeration of the manuscript’s parts. The goal is to precisely and concisely present the nature and scope of the research. An abstract should never include references or figures. If the thesis or the dissertation is in English, the résumé (French-language abstract) should come first followed by the abstract.

\end{otherlanguage}


\end{document}