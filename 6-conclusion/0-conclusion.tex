\documentclass[../Document.tex]{subfiles}
\graphicspath{{\subfix{../images/}}}

\begin{document}


\Chapter{CONCLUSION}
\label{chap:conclusion}
This chapter will conclude our work.
We will first go over the work we did, summarizing our main findings and their impact in regard to the research questions stated at the start of the paper.
Then we will highlight the limitations of our work and how it might have affected our results.
Finally, we will discuss future work that could be of interest to us our future researchers wishing to pursue this topic further.

%%
%%  SYNTHESE DES TRAVAUX / SUMMARY OF WORKS
%%
\section{Summary of Works}
\label{sec:conc/summary}
We defined a model capable of modelling valid molecules using a commonly used one-dimensional representation: \smiles.

We first test this model using additional structural constraints and three heuristics:\\ \texttt{domWdeg/Random}, \texttt{maxMarginalStrength/DFS}, and \texttt{maxMarginalStrength/LDS}.
On every instance, our model manages to find at least one solution across the three heuristics, confirming that our representation can generate valid molecules in \smiles format.

We then model desirable molecular properties using \cp and apply the constraints to our model, without the structural constraints.
This model is capable of generating valid molecules while targeting specific properties.
These property constraints are estimation with acceptable error rates.
Once again, the model is tested using the same three heuristics and every instance is resolved at least once.



\begin{enumerate}
    \item Can we use \acrshort{cp} to model valid molecules in a one-dimensional encoding?
    \item Can we use \acrshort{cp} to model desirable molecular properties in \acrshort{smiles} molecules?
    \item Can \acrlong{bp} be used to better guide a solver towards a solution?
    \item How can we combine a \acrshort{cp} model with a \acrshort{nlp} model to improve the realism of generated sequences and is it an effective method?
\end{enumerate}

%%
%%  LIMITATIONS
%%
\section{Limitations de la solution proposée / Limitations}
\label{sec:conc/limitations}

%%
%%  AMELIORATIONS FUTURES / FUTURE RESEARCH
%%
\section{Améliorations futures / Future Research}
\label{sec:conc/future}
Texte / Text.



\end{document}