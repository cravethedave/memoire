\documentclass[../Document.tex]{subfiles}
\graphicspath{{\subfix{../images/}}}

\begin{document}


\Chapter{LITERATURE REVIEW}
\label{chap:lit-review}

This chapter will focus on an overview of the current state of the art on this topic of research.

Drug discovery, and molecule design in general, is a vast topic.
There are many different methods that are applicable to the problem.

A recent survey by Du et al.~\cite{du2022molgensurvey} presents various representation formalisms. It covers one-dimensional representations such as \acrshort{smiles} and \acrshort{inchi} as well as two-dimensional and three-dimensional representations.
It describes some of the main problems tackled, and an array of computational methods used to solve them, mostly generative machine learning but also combinatorial solvers.


Among the current challenges for deep generative models, they mention the difficulty of exploring little known/seen areas of the molecular design space (the common out-of-distribution generation issue) and the need for lots of training data (generation in low-data regime issue i.e. high sample complexity).
They also mention as opportunity the generation of specialized molecules with more complex structure.

\section{\acrshort{nlp} applied to drug discovery}
\label{sec:lit-review/nlp}
% TODO Add the ones we inspired ourselves from and explain how they are relevant



\section{\acrshort{cp} applied to drug discovery}
\label{sec:lit-review/cp}



\subsection{Combining \acrshort{cp} with \acrshort{ml}}
\label{sec:lit-review/gpt+cp}
% TODO add Virasone, maybe Daphne


\end{document}